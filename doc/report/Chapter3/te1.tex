\documentclass[12 pt]{article}
\usepackage{amsmath}
\usepackage{graphicx}
\usepackage{latexsym}

\begin{document}
We start with a model considering a rover with two wheel, and with friction when contact happens.
We use Lagrange´s equation to create the differential equations system.
In this system the number of degree of freedom is 5, we use generalized coordinates $x,y,$ and $\theta$, $\theta_A$, $\theta_B$ to describe the system.

$x and y$ is the location of centre of gravity, and $\theta$ is the angle of the rob, $\theta_A$, $\theta_B$ are the angle of the wheel A and B receptively.


The kinetic Emery of rod is 
\begin{eqnarray}
\frac{1}{2}m_1(\dot{x}^2+\dot{y}^2)+\frac{1}{2}J_1\dot{\theta}^2
\end{eqnarray}


The velocity of the end of the rob is
Point A
\begin{eqnarray}
&v_{rollerA}=\sqrt{(l\dot{\theta}cos\theta+\dot{y})^2+(-l\dot{\theta}sin\theta+\dot{x})^2} \quad \text{Point A}&\\
&v_{rollerB}=\sqrt{(-l\dot{\theta}cos\theta+\dot{y})^2+(l\dot{\theta}sin\theta+\dot{x})^2} \quad \text{Point B}&
\end{eqnarray}

So the kinetic energy and rotation energy of roller is

\begin{eqnarray}
\frac{1}{2}m_2v_{roller}^2+\frac{1}{2}J_2\dot{\theta}^2
\end{eqnarray}

The total kinetic energy of the system is
\begin{equation}
\begin{aligned}
K=&\frac{1}{2}m_1(\dot{x}^2+\dot{y}^2)+\frac{1}{2}J_1\dot{\theta}^2\\
&+m_2(l^2\dot{\theta}^2+\dot{y}^2
+\dot{x}^2)+\frac{1}{2}J_2(\dot{\theta_A}^2+\dot{\theta_B}^2)
\label{k1}
\end{aligned}
\end{equation}

If we ignore the friction force, and set centre of roller1 is A,centre of roller2 is B and the gravity centre is O
the coordinates of these three points are:

\begin{eqnarray}
&(x_A,y_A)=(x-lcos\theta+rsin\theta_A,y-lsin\theta+rcos\theta_A)& \notag \\
&(x_B,y_B)=(x+lcos\theta+rsin\theta_B,y+lsin\theta+rcos\theta_B)& \notag \\
&(x_O,y_O)=(x,y)& \notag 
\end{eqnarray}

\begin{figure}[htbp]
\begin{center}
\input{pic1.pstex_t} 
\caption{pic1}
\label{figure:sdd}
\end{center}
\end{figure}

\begin{figure}[htbp]
\begin{center}
\input{pic2_t} 
\caption{pic2}
\label{figure:sdd}
\end{center}
\end{figure}

To compute the generilized force, we make a transformation for coordinate system and the equation \ref{k3a}
Take wheel A as an example.
\begin{equation}
Q_{k}=\sum_{i=1}^n (F_{xi}\frac{\partial x_{i}}{\partial q_k}+F_{y_i}\frac{\partial y_{i}}{\partial q_k}+F_{zi}\frac{\partial z_{i}}{\partial q_k})
\label{k3a}
\end{equation}

\begin{equation}
\begin{pmatrix}
F_Ax\\F_Ay
\end{pmatrix}=
\frac{1}{\sqrt{k_A^2+1}}\begin{pmatrix}
-k_A & 1\\
1 & k_A \end{pmatrix}
\begin{pmatrix}
\lambda_An \\ \lambda_At
\label{k2a}
\end{pmatrix}
\end{equation}

\begin{equation}
\begin{pmatrix}
R_{Ax}\\ R_{Ay}\\ R_{A\theta} \\R_{A\theta_A}\\ R_{A\theta_B}
\end{pmatrix}=\begin{pmatrix}
1 & 0 \\
0 & 1 \\
lsin\theta & -lcos\theta \\
rcos\theta_A & -rsin\theta_A \\
0 & 0 \end{pmatrix}
\begin{pmatrix}
F_{Ax} \\ F_{Ay}
\label{k1a}
\end{pmatrix}
\end{equation}

Substitude Equ \ref{k2a} into Equ \ref{k1a}
we have the relation from local to global coordinate.

\begin{equation}
\begin{pmatrix}
R_{Ax}\\ R_{Ay}\\ R_{A\theta} \\R_{A\theta_A}\\ R_{A\theta_B}
\end{pmatrix}=\frac{1}{\sqrt{k_A^2+1}}
\begin{pmatrix}
-k_A & 1 \\
1 & k_A \\
 -k_Alsin\theta-lcos\theta & lsin\theta-k_Alcos\theta \\
 -k_Arcos\theta_A-rsin\theta_A & rcos\theta_A-k_Arsin\theta_A\\
0& 0 \end{pmatrix}
\begin{pmatrix}
\lambda_An \\ \lambda_At
\end{pmatrix}
\end{equation}

Use similar procedure for wheel B
\begin{equation}
\begin{pmatrix}
R_{Bx}\\ R_{By}\\ R_{B\theta} \\R_{B\theta_A}\\ R_{B\theta_B}
\end{pmatrix}=\frac{1}{\sqrt{k_B^2+1}}
\begin{pmatrix}
 -k_B & 1\\
1 & k_B  \\
k_Blsin\theta+lcos\theta & -lsin\theta+k_Blcos\theta\\
0& 0 \\
-k_Brcos\theta_B-rsin\theta_B & rcos\theta_B-k_Brsin\theta_B
 \end{pmatrix}
\begin{pmatrix}
\lambda_Bn \\ \lambda_Bt
\end{pmatrix}
\end{equation}

So ,we can get the generalised force (without external force):

\begin{equation}
F=\begin{pmatrix}
0 \\ -m_1-2m_2 \\ 0 \\0 \\ 0 \end{pmatrix} +R_A+R_B
\end{equation}

We can have Lagrange equations as follow:
\begin{eqnarray}
&\frac{d}{dt}(\frac{\partial K}{\partial \dot{x}})-\frac{\partial K}{\partial x}=R_{Ax}+R_{Bx}& \label{k2} \\
&\frac{d}{dt}(\frac{\partial K}{\partial \dot{y}})-\frac{\partial K}{\partial y}=-m_1-2m_2+R_{Ay}+R_{By}& \label{k3} \\
&\frac{d}{dt}(\frac{\partial K}{\partial \dot{\theta}})-\frac{\partial K}{\partial \theta}=R_{A\theta}+R_{B\theta}& \\ \label{k4}
&\frac{d}{dt}(\frac{\partial K}{\partial \dot{\theta_A}})-\frac{\partial K}{\partial \theta_A}=R_{A\theta_A}+R_{B\theta_A}&\\
&\frac{d}{dt}(\frac{\partial K}{\partial \dot{\theta_B}})-\frac{\partial K}{\partial \theta_B}=R_{A\theta_B}+R_{B\theta_B}&\\
\end{eqnarray}

We define two functions $d_A(x,y,\theta), \quad d_B(x,y,\theta)$in the genernalized coordinates system to illustrate the nearest distance from the point A and B to the ground surface function $g(x)$

\begin{equation}
  \left\{
   \begin{aligned}
   &d_A(x,y,\theta)\ge 0\\
   &d_B(x,y,\theta)\ge0 \\
   R_A&=0 \quad \text{if $d_A(x,y,\theta)>0$}\\
   R_B&=0 \quad \text{if $d_B(x,y,\theta)>0$} \\
   R_A&>0 \quad \text{if $d_A(x,y,\theta)=0$}\\
   R_B&>0 \quad \text{if $d_B(x,y,\theta)=0$}\\
   f_A&=\mu R_A\\
   f_B&=\mu R_B 
   \end{aligned}
  \right.
  \end{equation}

Substitude equation\ref{k1} into equation \ref{k2},\ref{k3},\ref{k4}, we have 
\begin{eqnarray}
&m_1\ddot{x}=R_{Ax}+R_{Bx}& \\
&m_1\ddot{y}=-m_1g-2m_2g+R_{Ay}+R_{By}&\\
&J_1+2l^2m_2=R_{A\theta}+R_{B\theta}&\\
&J_2\ddot{\theta_A}=R_{A\theta_A}+R_{B\theta_A}&\\
&J_2\ddot{\theta_B}=R_{A\theta_B}+R_{B\theta_B}&
\end{eqnarray}

In matrix form:
\begin{equation}
q=\begin{pmatrix}
x\\y\\ \theta \\ \theta_A \\ \theta_B
\end{pmatrix} 
\end{equation}

\begin{equation}
\begin{pmatrix} m_1+2m_2 & 0 & 0 & 0 & 0\\
               0 & m_1+2m_2 & 0 & 0 & 0\\
               0 & 0 & J_1+2l^2m_2& 0 & 0\\
               0& 0& 0& J_2 & 0 \\
               0 & 0 & 0& 0& J_2
\end{pmatrix}\ddot{q}=F
\end{equation}

Relations:

\begin{eqnarray}
&h_1(q)=d_A(x,y,\theta)& \\
&h_2(1)=d_B(x,y,\theta)&\\
&H_1=\frac{1}{\sqrt{k_A^2+1}}
\begin{pmatrix}
-k_A &1 \\
1 & k_A \\
 -k_Alsin\theta-lcos\theta & lsin\theta-k_Alcos\theta \\
 -k_Arcos\theta_A-rsin\theta_A & rcos\theta_A-k_Arsin\theta_A\\
0& 0 \end{pmatrix}&\\
&H_2=\frac{1}{\sqrt{k_B^2+1}}
\begin{pmatrix}
 -k_B & 1\\
1 & k_B  \\
k_Blsin\theta+lcos\theta & -lsin\theta+k_Blcos\theta\\
0& 0 \\
-k_Brcos\theta_B-rsin\theta_B & rcos\theta_B-k_Brsin\theta_B
 \end{pmatrix}&
\end{eqnarray}

The relation between global velocity and local velocity can also obtained by using coordinate transformation.
Take the wheel A as an example.
\begin{equation}
\begin{pmatrix}
v_{xlocal}\\
v_{ylocal}
\end{pmatrix}=
\frac{1}{\sqrt{k^2+1}}\begin{pmatrix}
1 & k\\
-k & 1 
\end{pmatrix}
\begin{pmatrix}
1& 0 & lsin\theta & 0 & 0\\
0 & 1 & -lcos\theta & 0& 0
\end{pmatrix}
\begin{pmatrix}
\dot{x}\\ \dot{y}\\ \dot{\theta} \\ \dot{\theta}_A \\ \dot{\theta}_B
\end{pmatrix}+
\begin{pmatrix}
0 & 0 & 0 & R & 0
\end{pmatrix}
\begin{pmatrix}
\dot{x}\\ \dot{y}\\ \dot{\theta} \\ \dot{\theta}_A \\ \dot{\theta}_B
\end{pmatrix}
\end{equation}
\begin{equation}
\begin{pmatrix}
v_{ylocal}\\
v_{xlocal}
\end{pmatrix}=
\begin{pmatrix}
\frac{-k}{\sqrt{k^2+1}} & \frac{1}{\sqrt{k^2+1}} & \frac{-klsin\theta-lcos\theta}{\sqrt{k^2+1}} & 0 & 0\\
\frac{1}{\sqrt{k^2+1}} & \frac{k}{\sqrt{k^2+1}} & \frac{lsin\theta-klcos\theta}{\sqrt{k^2+1}} & R & 0\\
\end{pmatrix}
\begin{pmatrix}
\dot{x}\\ \dot{y}\\ \dot{\theta} \\ \dot{\theta}_A \\ \dot{\theta}_B
\end{pmatrix}
\end{equation}
\end{document}


