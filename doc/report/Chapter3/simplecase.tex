\documentclass[12 pt]{article}
\usepackage{amsmath}
\usepackage{graphicx}
\usepackage{latexsym}

\begin{document}

Consider the simple case to get used to the code.

\begin{figure}[htbp]
\begin{center}
\input{Chapter/simplecaseplot_t} 
\caption{Model}
\end{center}
\end{figure}


The kinetic energy is 
\begin{equation}
K=\frac{1}{2}m(\dot{x}^2+\dot{y}^2)+\frac{1}{2}J\dot{\theta}^2
\end{equation}


In the interaction point of the wheel and the ground, we have friction force $R_f$,and reaction force $R_n$
the coordinate is $(x+rsin\theta,y+rcos\theta)$

Make the coordinate transformation 
\begin{equation}
\begin{pmatrix}
F_x\\F_y
\end{pmatrix}=
\frac{1}{\sqrt{k^2+1}}\begin{pmatrix}
-k & 1\\
1 & k \end{pmatrix}
\begin{pmatrix}
\lambda_n \\ \lambda_t
\end{pmatrix}
\end{equation}

Generilized Force:
\begin{equation}
Q_k=F_x\frac{\partial x}{\partial q_k} + F_y\frac{\partial y}{\partial q_k}
\end{equation}

\begin{equation}
\begin{aligned}
\begin{pmatrix}
R_x \\ R_y \\R_{\theta}
\end{pmatrix}
=&\begin{pmatrix}
1 & 0 \\
0 & 1 \\
rcos\theta & -rsin\theta
\end{pmatrix}
\begin{pmatrix}
F_x\\F_y
\end{pmatrix}\\
=&\begin{pmatrix}
1 & 0 \\
0 & 1 \\
rcos\theta & -rsin\theta
\end{pmatrix}
\frac{1}{\sqrt{k^2+1}}\begin{pmatrix}
-k & 1\\
1 & k \end{pmatrix}
\begin{pmatrix}
\lambda_n \\ \lambda_t
\end{pmatrix}
\end{aligned}
\end{equation}
Calculate the generalized force, then we get Lagrangian equations:

\begin{eqnarray}
&\frac{d}{dt}(\frac{\partial K}{\partial \dot{x}})-\frac{\partial K}{\partial x}=R_x& \label{k2} \\
&\frac{d}{dt}(\frac{\partial K}{\partial \dot{y}})-\frac{\partial K}{\partial y}=-mg+R_y& \label{k3} \\
&\frac{d}{dt}(\frac{\partial K}{\partial \dot{\theta}})-\frac{\partial K}{\partial \theta}=R_{\theta}\label{k2} \\
\end{eqnarray}

\begin{eqnarray}
&m\ddot{x}=-R_x&\\
&m\ddot{y}=-mg+R_y&\\
&J\ddot{\theta}=R_{\theta}&
\end{eqnarray}

NewtonimpactFrictionNSL

\begin{equation}
y=[y_n,y_t]^T,\lambda=[\lambda_n,\lambda_t]^T
\end{equation}

\begin{equation}
if y_n=0, \left\{ \begin{matrix} 
0\le \dot{y}_n \bot \lambda_n \ge 0 \\
\dot{y}_t=o,\|\lambda_t \| \le \mu \lambda_n\\
\dot{y}_n \neq0, \lambda_t=-\mu \lambda_n sign(\dot{y}_t)
\end{matrix} \right.
\end{equation}

In matrix form:
\begin{equation}
\begin{pmatrix} m & 0 & 0 \\
0 & m & \\
0 & 0 & J \end{pmatrix}
\ddot{q}= \begin{pmatrix}
0 \\ -mg \\ 0 
\end{pmatrix}+R
\end{equation}

\begin{equation}
R=\begin{pmatrix}
R_x \\ R_y \\ R_{\theta}
\end{pmatrix}
\end{equation}

\begin{equation}
q=\begin{pmatrix} x \\y \\ \theta \end{pmatrix}
\end{equation}

\begin{equation}
R=H\begin{pmatrix}
\lambda_n
\\ \lambda_t \end{pmatrix}
\end{equation}


\begin{equation}
H=\frac{1}{\sqrt{k^2+1}}\begin{pmatrix}
-k & 1 \\
1 & k \\
-krcos\theta-rsin\theta & rcos\theta-krsin\theta
\end{pmatrix}
\end{equation}

\begin{equation}
y_{constrain}(d,y,\theta)=h(q)=y-kx-b-\frac{r\sqrt{k^2+1}}{k}
\end{equation}
\end{document}
