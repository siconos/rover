\chapter{Introduction of INRIA Rh\^one Alpes and Team Bipop}
\ifpdf
    \graphicspath{{Chapter1/Chapter1Figs/PNG/}{Chapter1/Chapter1Figs/PDF/}{Chapter1/Chapter1Figs/}}
\else
    \graphicspath{{Chapter1/Chapter1Figs/EPS/}{Chapter1/Chapter1Figs/}}
\fi


\section{INRIA Rh\^one Alpes}

INRIA Grenoble - Rh\^one-Alpes Research Centre is one of the eight research centres run by INRIA, the French National Institute for Research in Computer Science and Control. As a public institute jointly supervised by the French Ministries of Research and of Industry, the centre aims to fulfil two key duties:

\begin{itemize}
\item To carry out fundamental and applied research in information and communication science and technology (ICST).
\item To share its research results and promote their application in society.
\end{itemize}

And there are some useful key points for INRIA Rh\^one Alpes:

\begin{itemize}
\item  Founded in: 1992
\item Approximately 600 people, half of whom are on INRIA's payroll and a third of whom are doctoral students
\item 28 research teams
\item Research support departments
\item Approximately 50\% of its budget comes from contracted project funding.
\end{itemize}

\section{Team Bipop}

Team Bipop is concerned with non-smooth dynamical systems and non-smooth optimization. More precisely, modelling, control and numerical simulation are the main scientific topics. The basic tools therefore come from non-smooth mechanics, systems and control theory, non-smooth optimisation, and convex and non-smooth analysis.\\



The main applications can be found in mechanical systems (multibody systems with unilateral constraints, friction, nonsmooth contact laws), and in electrical systems (circuits with diodes, MOS transistors). Some more abstract problems (like optimal control with state constraints, generalized predictive control) also fit within this framework.\\

The main areas of application are: automotive systems, aerospace applications, electro-mechanical systems (mechatronics), robotics, etc. There are still many open fields of theoretical research (in systems theory: controllability, observability, stabilisation, trajectory tracking; in mechanical modelling: multiple impacts modelling, nonmonotone contact laws, Painlevé paradoxes), as well as on a more applied level (numerical simulation and software development). The biped robot is one of the many applications for control and simulation.\\


\section{Short introduction of the Siconos Platform}

The Siconos Platform is a scientific computing software dedicated to the modeling, simulation, control and analysis of Non Smooth Dynamical Systems (NSDS). Especially, the following classes of NSDS are addressed:

\begin{itemize}
\item Mechanical systems with contact, impact and friction
\item Electrical circuits with ideal and piecewise linear components
\item Differential inclusions and Complementarity systems
\end{itemize}






